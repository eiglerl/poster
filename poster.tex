
% POSTER EXAMPLE
%
% This is an example of a relatively sane poster. The box structure (and the
% narrative in general) is what I would expect, but it is completely
% non-mandatory; you may include whatever you want. Preferably, erase the
% existing box structure after you read it, and start from scratch.
%
% The main communication requirements for the poster that should be satisfied
% are as such:
%
% - At the defense, it should help you talk for around 10 minutes about your
%   thesis, and convince the committee that you did something interesting and
%   sufficiently complicated. Prepare pictures that explain your main results.
%
% - It should quickly communicate the main idea of your thesis to a random
%   educated by-walker. Ideally, a moderately-witted MFF graduate who has never
%   heard about your thesis before should be able to get the main "rough idea"
%   in less than 1 minute by just looking at the poster.

% modify the fontscale parameter to make everything slighly bigger or smaller.
\documentclass[portrait,a0paper,fontscale=0.25]{baposter}

\usepackage[utf8]{inputenc}
\usepackage[T1]{fontenc}

% FONT CHOICES
% Posters do not need to be PDF/A; you can choose any relatable font from the
% TeX font catalogue without much risk. Sans-serif fonts are suggested for the
% posters; see https://tug.org/FontCatalogue/sansseriffonts.html
\usepackage[sfdefault]{Fira Sans}
%\usepackage[default]{droidsans}
%\usepackage[math]{iwona}
%\usepackage[defaultfam]{montserrat}
%\usepackage{cmbright}
%\usepackage{yfonts}\renewcommand{\familydefault}{\frakdefault}

\usepackage{color}
\usepackage{graphicx}
\usepackage{amssymb,amsmath}
\usepackage[export]{adjustbox} %allows using valign with \includegraphics

\renewcommand{\arraystretch}{1.5}

\usetikzlibrary{positioning}

% A WORD ABOUT COLORS
%
% This template is prepared with a relatively neutral gray background that
% gives decent box borders (with white and darker gray), does not clash with
% many colors (except for violet-brown and other mushroomish colors, perhaps)
% and gives a lot of space for highlighting stuff.
%
% Generally, other color variations are good too; there are no strict rules on
% the colors. Good choices include:
%
% - white backgrounds and differentiation of box headers by color (see
%   headerFontColor)
%
% - various slightly tinted backgrounds (try red!10 instead of black!3)
% 
% - dark backgrounds
%
% Keep in mind:
% - The normal "informative" text and figures should be DARK on LIGHT
%   background, not the other way around.
%
% - If you want a dark background, soften (darken) the box backgrounds a bit so
%   that they do not "shine" too much from the poster. Use \color{white} for
%   the heading, and switch the UK/MFF logos to white (see contents of logos/).
%
% - Do not mix too many color hues together. Most hues have their widely
%   accepted meaning (green: good result, red: problem, blue: information,
%   yellow: highlighter, brown: serious problem, violet: something really
%   weird/interesting/magic, depending on the shade).

\begin{document}

\color{black!80} % default font color
\begin{poster}{grid=false,
	eyecatcher=true,
	background=plain,
	bgColorOne=black!3, % background color
	columns=2,
	headerborder=none,
	textborder=none,
	headershape=rectangle,
	headershade=plain,
	boxshade=plain,
	boxColorOne=white,
	headershade=plain,
	headerColorOne=black!15, % box header background color
	headerFontColor=black,
	}%
	{\includegraphics[height=7em]{logos/mff-black.pdf}}
	{Umělá inteligence pro strategické hry s~neúplnou informací}
	{\vspace{1ex} Lukáš Eigler}
	{\includegraphics[height=7em]{logos/uk-red.pdf}}


%
% LEFT COLUMN
%

\begin{posterbox}[column=0,name=background]{Úvod}
Tato práce řeší hru Fantom staré Prahy. Hrají proti sobě dva týmy, Fantom (jeden hráč) a detektivové (pět hráčů). Cíl Fantoma je nebýt dopaden po dobu hry - 24 kol, naopak detektivové mají za úkol Fantoma dopadnout. Hráči se pohybují po mapě zastávek spojených cestami dopravních prostředků. Pro přesun pomocí dopravního prostředku je nutné mít správný žeton. Háček je, že se Fantom pohybuje neviditelný a ukáže se pouze v pár předem určených kolech. Poskytuje pouze informaci o použitém dopravním prostředku.

\begin{center}
\includegraphics[width=0.7\linewidth]{img/ingame_view.png}
\end{center}

%\begin{center}\begin{tikzpicture}[ultra thick, inner sep=1ex]
%\node[rectangle, rounded corners=1ex, draw=red!80!black, color=red!80!black, font=\huge\bfseries, rotate=21] {Problem!};
%\end{tikzpicture}\end{center}
\end{posterbox}

%\begin{posterbox}[column=0, name=goals, below=background, headerColorOne=cyan!60, boxColorOne=cyan!20]{Cíl práce}
\begin{posterbox}[column=0, name=goals, below=background]{Cíl práce}
Mezi cíle této práce patří naprogramovat a připravit hratelnou implementaci deskové hry "Fantom staré Prahy" pomocí herního engine Unity. Hra poskytuje možnost doplnění o vlastní implementaci umělého agenta, jak za Fantoma či detektivy. Hlavním cílem práce bylo připravit agenty s různou úrovní obtížnosti pro případné hraní. 

\end{posterbox}

\begin{posterbox}[column=0, name=something1, below=goals]{Řešení}
Díky vlastnostem hry se jedná o hru s neúplnou informací - detektivové většinu kol neznají přímou polohu Fantoma. Pro vymyšlení umělého hráče proto využíváme některé teoretické poznatky z teorie her, což je obor matematiky/ekonomie. 
Připravili jsme dva druhy agentů. První využívá heuristiky pro rozhodování a druhý Monte Carlo metody. Zvolené heuristiky pro každý tah ohodnotí možné pohyby dle vzdálenosti od detektivů, respektive od Fantoma, a zvolí ten nejvzdálenější pro Fantoma, respektive nejbližší pro detektivy. Oproti tomu agent využívající Monte Carlo metody má obecný Information Set Monte Carlo Tree Search (ISMCTS) algoritmus pro prohledávání prostoru. Samostatný Monte Carlo Tree Search (MCTS) není dostačující kvůli neúplné informaci, ISMCTS problém řeší pomocí tzv. determinizace - náhodné zvolení možného stavu a následné průměrování přes všechny stavy. Přestože má determinizace teoretické nedostatky, na některé problémy se dá využít. Hráč s ISMCT kromě popisu hry nevyužívá žádné znalosti domény.

\begin{center}
\includegraphics[width=1\linewidth]{img/mcts-alg.png}
\end{center}


%This box may contain an overview of the used methods, mathematics, program structure, etc.

%Include a picture, because pictures are better. Thesis defense takes less than 10 minutes, no one can read a wall of text in that short time. (For comparison, the usual realistic poster visit on a conference takes 15 seconds, unless the poster manages to catch the attention in that short time.)

%\tikzstyle{rec}=[rectangle, draw, rounded corners=1ex, font=\huge\bfseries]
%\begin{center}\begin{tikzpicture}[ultra thick, inner sep=1ex]
%\node[rec] (a) {Keep};
%\node[rec, circle, right=of a] (b) {it};
%\node[rec, right=of b] (c) {simple};
%\node[rec, densely dotted, below=6cm of c, font=\small] (notice) {\dots{}but precise!};
%\draw[->] (a) to (b);
%\draw[->] (b) to (c);
%\draw[dotted] (notice) to (c);
%\end{tikzpicture}\end{center}

\end{posterbox}

%\begin{posterbox}[column=0, name=something2, below=something1, headerColorOne=yellow!80!orange!95!black, boxColorOne=yellow!33]{Golden rule of posters}
%For each $p \in P$ where $P$ is a set of posters:
%$$ \textsc{Success}(p) = \frac{\textsc{Clarity}(p)}{\textsc{TimeToViewerAttention}(p)} $$
%\end{posterbox}

%
% FOOTER
%

%\begin{posterbox}[column=0, span=2, name=footer, below=something2,
%	textborder=none, headerborder=none, boxheaderheight=0pt,
%	boxColorOne=black!3]{}
%If some institute/grant/department sponsored the work, put an acknowledgement here.
%\end{posterbox}

%
% RIGHT COLUMN
%
% It is usually best to fill most of the poster with your results and
% conclusions. Again, use simple annotated pictures wherever possible. Plots
% with measurements are perfect, tables are also good.
%

\begin{posterbox}[column=1, name=result1]{Výsledky}
Porovnání kvality umělých agentů probíhalo jako menší turnaj na 3 různých náhodně vygenerovaných mapách. Na každé mapě se zápas opakoval 50krát. Nejdříve se hráč využívající heuristiku utkal proti náhodnému hráči. Vítěz, v obou případech hráč s heuristikou, se utkal s proti agentovi s ISMCTS. Tabulky níže popisují výhernost hráče využívající heuristiku proti hráči využívající ISMCTS.  
V první tabulce využívá Fantom ISMCTS a detektivové heuristiku.
\begin{center}
\includegraphics[width=0.7\linewidth]{img/fantom-mcts.png}
\end{center}

Ve druhé tabulce naopak Fantom využívá heuristiku a detektivové ISMCTS.
\begin{center}
\includegraphics[width=0.7\linewidth]{img/fantom-mcts.png}
\end{center}

Histogram délek her, kdy vyhráli detektivové v zápasech, kdy Fantom využíval ISMCTS a detektivové heuristiku.
\begin{center}
\includegraphics[width=0.7\linewidth]{img/fantom-mcts.png}
\end{center}




\end{posterbox}

\begin{posterbox}[column=1, name=result2, below=result1]{Závěr}
Práce splnila předem dané cíle. Implementace hry nabízí možnost za Fantoma i detektivy zvolit reálného i umělého hráče, umožňuje tedy hrát lidským hráčům proti sobě, ale i proti různě silným soupeřům umělé inteligence.
\end{posterbox}

\begin{posterbox}[column=1, name=result3, below=result2]{Možná vylepšení}
Implementace algoritmu ISMCTS nabízí několik možností vylepšení. Pro kvalitnější strategie je možné zabudovat do algoritmu, například zvolením chytřejší rollout strategie místo uniformní, která lépe reprezentuje chování racionálního hráče. Jiným možným vylepšením je vyřešit problém neúplné informace lepším způsobem, například nevybírat stav uniformě, ale dle pravděpodobnostní distribuce aktuální přes možné pozice Fantoma.  
  
Chytřejší agent se dá získat i pomocí konceptu regret z teorie her. Tohoto konceptu využívá algoritmus Student of Games, který je schopnýn hrát obecnou hru s neúplnou informací na rozumné úrovni, popsaný v práci.
\end{posterbox}



\end{poster}
\end{document}
